\documentclass[12pt, a4paper]{article}
\usepackage[utf8]{inputenc}
\usepackage[T1]{fontenc}
\usepackage{amsthm}
\usepackage{algorithm2e}
\usepackage{multirow}
\usepackage{adjustbox}

\theoremstyle{definition}
\newtheorem{definition}{Definicija}[section]

\title{Heurističko rešavanje problema minimalnog broja zadovoljivih formula}
\author{
	Aleksa Papić
	\and
	Aleksandar Stefanović
}
\date{17. septembar 2022.}

\renewcommand*\contentsname{Sadržaj}
\renewcommand*\tablename{Tabela}
\renewcommand*\figurename{Slika}
\renewcommand*\refname{Reference}
\renewcommand*\algorithmcfname{Algoritam}

\sloppy

\RestyleAlgo{ruled}

\begin{document}

\maketitle

\tableofcontents

\newpage

\section{Uvod}

Problem minimalne zadovoljivosti iskazne formule $f$ (eng. MIN-SAT) je optimizaciona varijanta problema zadovoljivosti (eng. SAT) u kojoj se traži valuacija $v$ takva da je broj klauza formule $f$ tačnih u valuaciji $v$ minimalan. Poznato je da je ovaj problem NP-težak \cite{1}.

U ovom radu ćemo posmatrati naredno modifikaciju ovog problema datu u \cite{2}:
\begin{definition}[Problem minimalnog broja zadovoljivih formula]
\label{def 1.1}
Neka je dat par $(U, C)$ gde je $U$ skup iskaznih promenljivih, a $C$ skup iskaznih formula u 3KNF (eng. 3CNF). Rešenje problema minimalnog broja zadovoljivih formula nad $(U, C)$ je valuacija $v$ za promenljive iz skupa $U$ takva da je broj formula iz skupa $C$ zadovoljenih tom valuacijom minimalan.
\end{definition}

Iz definicije se može zaključiti da svaka instanca MIN-SAT problema odgovara nekoj instanci problema \ref{def 1.1} u kojoj je broj klauza svake formule iz $C$ jednak jedan.

Razmotrićemo i uporediti performanse jednog genetskog algoritma i više varijanti memetičkih algoritama za rešavanje problema \ref{def 1.1}.

\section{Opis algoritama}

U ovom poglavlju ćemo dati opis nekoliko pristupa u rešavanju problema \ref{def 1.1}.

\subsection{Kodiranje jedinki}

Pre razmatranja konkretnih algoritama, opisaćemo način kodiranja jedinki, tj. način predstavljanja konkretnih valuacija u okviru problema \ref{def 1.1}.

\begin{definition}[Kodiranje jedinki]
\label{def 2.1}
Neka je dat par $(U, C)$ kao u \ref{def 1.1} i neka je skup promenljivih $U = \{p_1, ..., p_n\}$. Tada je valuacija $v$ nad skupom promenljivih $U$ niz binarnih brojeva $(x_1, ..., x_n) \in \{0, 1\}^n$ takav da $x_i$ odgovara konkretizovanoj vrednosti promenljive $p_i$.
\end{definition}

\subsection{Ocena kvaliteta jedinki}

Ocenu kvaliteta jedinki u okviru problema \ref{def 1.1} ćemo zadati preko tzv. fitnes funkcije jedinke.

\begin{definition}[Fitnes funkcija]
\label{def 2.2}
Neka je $v$ neka jedinka, tj. konkretna valuacija (\ref{def 2.1}) za par $(U, C)$ definisan kao u problemu \ref{def 1.1}. Fitnes funkcija $fitness : \{0, 1\}^n \rightarrow (0, 1]$ je zadata sa $fitness(v) = \frac{1}{sat(v) + 1}$, gde je $sat(v)$ broj iskaznih formula iz $C$ zadovoljenih u valuaciji $v$.
\end{definition}

Iz date definicije se može zaključiti da je jedinka $v_1$ bolja od jedinke $v_2$ u kontekstu problema \ref{def 1.1} ako i samo ako je $fitness(v_1) > fitness(v_2)$.

Broj zadovoljenih formula u valuaciji $v$ se može dobiti kao $sat(v) = \frac{1}{fitness(v)} - 1$, ali se zbog računa u pokretnom zarezu predlaže zaokruživanje na najbliži ceo broj, tj.
$sat(v) = round(\frac{1}{fitness(v)} - 1)$.

\subsection{Rešavanje algoritmom grube sile}

Naivni algoritam kojim se problem rešava grubom silom proverava sve moguće valuacije u problemu. Ukoliko je $U$ skup promenljivih u problemu \ref{def 1.1}, takvih valuacija je $2^{|U|}$. Iako je egzaktan, ovaj algoritam je praktično neprimenljiv za sve osim najmanje probleme zbog svoje eksponencijalne složenosti.

\subsection{Rešavanje genetskim algoritmom}

Prvi heuristički algoritam koji ćemo razmotriti je genetski algoritam. Opisaćemo operatore koje ćemo koristiti, kao i uslov zaustavljanja. Detaljan opis algoritma i mogućih modifikacija se može videti u \cite{3}.

\begin{algorithm}
\caption{Genetski algoritam}
\label{alg:1}
$t \gets 0$\;
$P_0 \gets generisi\ populaciju()$\;
\While{nije ispunjen uslov zaustavljanja}{
	$P_{sel} \gets selekcija(P_t)$\;
	$P_{t + 1} \gets ukrstanje(P_{sel})$\;
	$P_{t + 1} \gets mutacija(P_{t + 1})$\;
	$t \gets t + 1$\;
}
\end{algorithm}

\subsubsection{Operator selekcije}

Za selekciju jedinki za reprodukciju koristićemo ruletsku selekciju. Verovatnoća izbora neke jedinke $v$ jednaka je $\frac{fitness(v)}{\sum_{u \in P_t} fitness(u)}$, gde je $P_t$ populacija jedinki u tekućoj iteraciji algoritma \ref{alg:1}.

\subsubsection{Operator ukrštanja}

Za ukrštanje jedinki prilikom reprodukcije koristićemo jednopoziciono ukrštanje. Dve jedinke, $r_1 = (x_1, ..., x_n)$ i $r_2 = (y_1, ..., y_n)$, izabrane za roditelje u fazi selekcije kreiraće dva potomka, $p_1$ i $p_2$, izborom \emph{tačke preseka} $k$ iz diskretne uniformne raspodele nad vrednostima $\{1, ..., n\}$. Tada će važiti $p_1 = (x_1, ..., x_k, y_{k + 1}, ..., y_n)$ i $p_2 = (y_1, ..., y_k, x_{k + 1}, ..., x_n)$.

\subsubsection{Operator mutacije}

Operator mutacije koji ćemo koristiti sa nekom verovatnoćom $p \in U(0, 1)$, koja se zadaje kao parametar algoritma \ref{alg:1}, invertuje jedan od bitova jedinke nad kojom se sprovodi mutacija.

\subsubsection{Uslov zaustavljanja}

Kao uslov zaustavljanja koristićemo maksimalni broj iteracija algoritma \ref{alg:1}, kao i maksimalni broj iteracija bez promene u najboljoj jedinki. Obe vrednosti se zadaju kao parametri algoritma.

\subsection{Rešavanje memetskim algoritmom}

Još jedan tip algoritama koje ćemo razmotriti su memetski algoritmi. Ovi algoritmi predstavljaju kombinaciju više različitih heurističkih pristupa rešavanju problema \cite{4}.

Konkretna implementacija koju ćemo razmotriti kombinuje genetski algoritam opisan u prethodnom poglavlju sa nekom S-metaheuristikom, a mi ćemo ih obraditi tri. Sledi uopšteni algoritam:

\begin{algorithm}
\caption{Memetski algoritam}
\label{alg:2}
$t \gets 0$\;
$P_0 \gets generisi\ populaciju()$\;
\While{nije ispunjen uslov zaustavljanja}{
	$P_{sel} \gets selekcija(P_t)$\;
	$P_{t + 1} \gets ukrstanje(P_{sel})$\;
	$P_{t + 1} \gets mutacija(P_{t + 1})$\;
	$P_{t + 1} \gets optimizacija(P_{t + 1})$\;
	$t \gets t + 1$\;
}
\end{algorithm}

Jedina razlika u odnosu na genetski algoritam \ref{alg:1} je novouvedeni operator optimizacije. Svi ostali operatori su implementirani identično kao u prethodnom poglavlju.

\subsubsection{Operator optimizacije}

Uloga operatora optimizacije je da se potencijalno poboljša svaka pojedinačna jedinka iz novonastale populacije. Ovo se suštinski postiže pozivanjem neke S-metaheuristike, koja je zadata kao parametar algoritma, nad svakom jedinkom.

\paragraph{Lokalna pretraga kao operator optimizacije}

Najjednostavnija S-metaheuristika koju ćemo koristiti je lokalna pretraga. Kriterijum zaustavljanja je broj iteracija koji se prosleđuje kao parametar algoritmu.

\begin{algorithm}
\caption{Lokalna pretraga}
\label{alg:3}
\KwData{$pocetna\ jedinka$}
\KwResult{$najbolja\ jedinka$}
$trenutna\ jedinka \gets pocetna\ jedinka$\;
\While{nije ispunjen uslov zaustavljanja}{
	$nova\ jedinka \gets invertuj(trenutna\ jedinka)$\;
	\If{fitness(nova jedinka) > fitness(trenutna jedinka)}{
		$trenutna\ jedinka \gets nova\ jedinka$\;
	}
}
$najbolja\ jedinka \gets trenutna\ jedinka$\;
\end{algorithm}

\paragraph{Simulirano kaljenje kao operator optimizacije}

Verovatnoća kaljenja se računa po formuli $\frac{1}{t^s}$, gde se broj $s$ prosleđuje kao parametar algoritma. Kriterijum zaustavljanja je broj iteracija koji se takođe prosleđuje kao parametar.

\begin{algorithm}
\caption{Simulirano kaljenje}
\label{alg:4}
\KwData{$pocetna\ jedinka$}
\KwResult{$najbolja\ jedinka$}
$trenutna\ jedinka \gets pocetna\ jedinka$\;
$najbolja\ jedinka \gets pocetna\ jedinka$\;
$t \gets 1$\;
\While{nije ispunjen uslov zaustavljanja}{
	$nova\ jedinka \gets invertuj(trenutna\ jedinka)$\;
	\eIf{fitness(nova jedinka) > fitness(trenutna jedinka)}{
		$trenutna\ jedinka \gets nova\ jedinka$\;
		\If{fitness(nova jedinka) > fitness(najbolja jedinka)}{
			$najbolja\ jedinka \gets nova\ jedinka$\;		
		}
	}{
		$p \gets \frac{1}{t^s}$\;
		\If{$q \in U(0, 1) < p$}{
			$trenutna\ jedinka \gets nova\ jedinka$\;
		}
	}
	$t \gets t + 1$\;
}
\end{algorithm}

\paragraph{Redukovana metoda promenljivih okolina kao operator optimizacije}

Okolina veličine $k$ neke jedinke $v$ predstavlja skup jedinki koje se mogu dobiti invertovanjem tačno $k$ bitova u reprezentaciji jedinke $v$. Maksimalna veličina okoline se prosleđuje kao parametar algoritma, kao i broj iteracija algoritma koji predstavlja kriterijum zaustavljanja.

\begin{algorithm}
\caption{Redukovana metoda promenljivih okolina}
\label{alg:5}
\KwData{$pocetna\ jedinka$}
\KwResult{$najbolja\ jedinka$}
$trenutna\ jedinka \gets pocetna\ jedinka$\;
\While{nije ispunjen uslov zaustavljanja}{
	\For{$k \gets 1$ \KwTo $maks.\ okolina$}{
		$nova\ jedinka \gets invertuj(trenutna\ jedinka, k)$\;
		\If{fitness(nova jedinka) > fitness(trenutna jedinka)}{
			$trenutna\ jedinka \gets nova\ jedinka$\;
			\textbf{break}\;
		}
	}
}
$najbolja\ jedinka \gets trenutna\ jedinka$\;
\end{algorithm}

\section{Rezultati}

\begin{table}
\begin{adjustbox}{width=\columnwidth,center}
\begin{tabular}{ |c|c|c|c|c|c|c| }
\hline
Instanca & small1-50-7-1 & small2-50-7-2 & small3-50-10-1 & small4-50-10-3 & small5-50-15-1 & small6-50-15-4 \\
\hline
|U| & 7 & 7 & 10 & 10 & 15 & 15 \\
\hline
|C| & 50 & 50 & 50 & 50 & 50 & 50 \\
\hline
BF opt. & 40 & 29 & 37 & 23 & 34 & 17 \\
\hline
GA najbolje & 40 & 29 & 37 & 23 & 34 & 17 \\
\hline
GA prosek & 40.0 & 29.0 & 37.17 & 23.1 & 34.38 & 18.37 \\
\hline
GA najgore & 40 & 29 & 39 & 27 & 36 & 21 \\
\hline
MA(LS) najbolje & 40 & 29 & 37 & 23 & 34 & 17 \\
\hline
MA(LS) prosek & 40 & 29.0 & 37.03 & 23.0 & 34.43 & 18.19 \\
\hline
MA(LS) najgore & 40 & 29 & 38 & 23 & 36 & 20 \\
\hline
MA(SA) najbolje & 40 & 29 & 37 & 23 & 34 & 17 \\
\hline
MA(SA) prosek & 40.0 & 29.0 & 37.01 & 23.0 & 34.56 & 18.17 \\
\hline
MA(SA) najgore & 40 & 29 & 38 & 23 & 36 & 20 \\
\hline
MA(RVNS) najbolje & 40 & 29 & 37 & 23 & 34 & 17 \\
\hline
MA(RVNS) prosek & 40.0 & 29.0 & 37.0 & 23.0 & 34.34 & 17.83 \\
\hline
MA(RVNS) najgore & 40 & 29 & 37 & 23 & 35 & 19 \\
\hline
t BF & < 0.01 & 0.01 & 0.04 & 0.07 & 1.18 & 2.70 \\
\hline
t GA & 0.03 & 0.06 & 0.05 & 0.08 & 0.05 & 0.11 \\
\hline
t MA(LS) & 0.11 & 0.16 & 0.13 & 0.24 & 0.15 & 0.34 \\
\hline
t MA(SA) & 0.11 & 0.16 & 0.14 & 0.24 & 0.16 & 0.34 \\
\hline
t MA(RVNS) & 0.24 & 0.35 & 0.26 & 0.46 & 0.32 & 0.72 \\
\hline
\end{tabular}
\end{adjustbox}
\end{table}

\section{Zaključak}

\newpage

\begin{thebibliography}{9}
\bibitem{1}
Rajeev Kohlit, Ramesh Krishnamurti, Prakash Mirchandani, \emph{The minimum satisfiability problem}. SIAM J. Discrete Math. Vol. 7, No. 2, pp. 275-283, May 1994.
\bibitem{2}
Viggo Kann, \emph{Polynomially bounded minimization problems that are hard to approximate}. Nordic Journal of Computing 1(1994), 317–331.
\bibitem{3}
Engelbrecht, Andries P. \emph{Computational intelligence : an introduction / Andries P. Engelbrecht. – 2nd ed.}
\bibitem{4}
Pablo Moscato, Carlos Cotta, Alexandre Mendes, \emph{Memetic Algorithms}. 
\end{thebibliography}

\end{document}