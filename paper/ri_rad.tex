\documentclass[12pt, a4paper]{article}
\usepackage[utf8]{inputenc}
\usepackage[T1]{fontenc}
\usepackage{amsthm}

\theoremstyle{definition}
\newtheorem{definition}{Definicija}[section]

\title{Heurističko rešavanje problema minimalnog broja zadovoljivih formula}
\author{
	Aleksa Papić
	\and
	Aleksandar Stefanović
}
\date{17. septembar 2022.}

\renewcommand*\contentsname{Sadržaj}
\renewcommand*\tablename{Tabela}
\renewcommand*\figurename{Slika}
\renewcommand*\refname{Reference}

\begin{document}

\maketitle

\tableofcontents

\section{Uvod}

Problem minimalne zadovoljivosti iskazne formule $f$ (eng. MIN-SAT) je optimizaciona varijanta problema zadovoljivosti (eng. SAT) u kojoj se traži valuacija $v$ takva da je broj klauza formule $f$ tačnih u valuaciji $v$ minimalan. Poznato je da je ovaj problem NP-težak \cite{1}.

U ovom radu ćemo posmatrati naredno modifikaciju ovog problema datu u \cite{2}:
\begin{definition}[Problem minimalnog broja zadovoljivih formula]
\label{def 1.1}
Neka je dat par $(U, C)$ gde je $U$ skup iskaznih promenljivih, a $C$ skup iskaznih formula u 3KNF (eng. 3CNF). Rešenje problema minimalnog broja zadovoljivih formula nad $(U, C)$ je valuacija $v$ za promenljive iz skupa $U$ takva da je broj formula iz skupa $C$ zadovoljenih tom valuacijom minimalan.
\end{definition}

Iz definicije se može zaključiti da svaka instanca MIN-SAT problema odgovara nekoj instanci problema \ref{def 1.1} u kojoj je broj klauza svake formule iz $C$ jednak jedan.

Razmotrićemo i uporediti performanse jednog genetskog algoritma i više varijanti memetičkih algoritama za rešavanje problema \ref{def 1.1}.

\section{Opis algoritama}

U ovom poglavlju ćemo dati opis nekoliko pristupa u rešavanju problema \ref{def 1.1}.

\subsection{Kodiranje jedinki}

Pre razmatranja konkretnih algoritama, opisaćemo način kodiranja jedinki, tj. način predstavljanja konkretnih valuacija u okviru problema \ref{def 1.1}.

\begin{definition}[Kodiranje jedinki]
\label{def 2.1}
Neka je dat par $(U, C)$ kao u \ref{def 1.1} i neka je skup promenljivih $U = \{p_1, ..., p_n\}$. Tada je valuacija $v$ nad skupom promenljivih $U$ niz binarnih brojeva $(x_1, ..., x_n) \in \{0, 1\}^n$ takav da $x_i$ odgovara konkretizovanoj vrednosti promenljive $p_i$.
\end{definition}

\subsection{Ocena kvaliteta jedinki}

Ocenu kvaliteta jedinki u okviru problema \ref{def 1.1} ćemo zadati preko tzv. fitnes funkcije jedinke.

\begin{definition}[Fitnes funkcija]
\label{def 2.2}
Neka je $v$ neka jedinka, tj. konkretna valuacija (\ref{def 2.1}) za par $(U, C)$ definisan kao u problemu \ref{def 1.1}. Fitnes funkcija $fitness : \{0, 1\}^n \rightarrow (0, 1]$ je zadata sa $fitness(v) = \frac{1}{sat(v) + 1}$, gde je $sat(v)$ broj iskaznih formula iz $C$ zadovoljenih u valuaciji $v$.
\end{definition}

Iz date definicije se može zaključiti da je jedinka $v_1$ bolja od jedinke $v_2$ u kontekstu problema \ref{def 1.1} ako i samo ako je $fitness(v_1) > fitness(v_2)$.

Broj zadovoljenih formula u valuaciji $v$ se može dobiti kao $sat(v) = \frac{1}{fitness(v)} - 1$, ali se zbog računa u pokretnom zarezu predlaže zaokruživanje na najbliži ceo broj, tj.
$sat(v) = round(\frac{1}{fitness(v)} - 1)$.

\subsection{Rešavanje grubom silom}

\subsection{Rešavanje genetskim algoritmom}

\subsection{Rešavanje memetskim algoritmom}

\begin{thebibliography}{9}
\bibitem{1}
Rajeev Kohlit, Ramesh Krishnamurti, Prakash Mirchandani, \emph{The minimum satisfiability problem}. SIAM J. Discrete Math. Vol. 7, No. 2, pp. 275-283, May 1994.
\bibitem{2}
Viggo Kann, \emph{Polynomially bounded minimization problems that are hard to approximate}. Nordic Journal of Computing 1(1994), 317–331.
\end{thebibliography}

\end{document}